% Using the RPI resume template found at
% https://www.rpi.edu/dept/arc/training/latex/resumes/
\documentclass[margin]{rpires}
\usepackage{amsfonts}
\usepackage{hyperref}
\setlength{\oddsidemargin}{-0.3in}
\setlength{\evensidemargin}{-0.3in}
\setlength{\topmargin}{-0.5in}
\setlength{\resumewidth}{7in}
\setlength{\textheight}{10in}
\newsectionwidth{1.4in}
\begin{document}  
% Center the name over the entire width of resume:
\moveleft.5\hoffset\centerline{\large\bf Matthew C. Pharr}
% Draw a horizontal line the whole width of resume:
 \moveleft\hoffset\vbox{\hrule width\resumewidth height 1pt}\smallskip
% address begins here
% Again, the address lines must be centered over entire width of resume:
 \moveleft.5\hoffset\centerline{matthew.pharr@columbia.edu}
 \moveleft.5\hoffset\centerline{(410) 375-9882}
                        
\begin{resume}                       

% \section{OBJECTIVE}       Acceptance to PhD Programs in Plasma Physics and Applied Mathematics.
 
\section{EDUCATION}      

Columbia University, New York, NY\\
Ph.D. Plasma Physics, Expected 2026 \\
M.Phil. Plasma Physics, Expected 2025\\
M.S. Applied Physics, Expected 2022 \\

\vspace{-5mm}
Rensselaer Polytechnic Institute, Troy, NY\\
B.S. Physics \&  Mathematics, 2021 \\
{\sl Summa Cum Laude.}

\section{PUBLICATIONS, TALKS, AND PROCEEDINGS}

% \begin{itemize}

% \item Pharr M, Ebrahimi F. {\sl Azimuthal Curvature Instability in Hall-MHD plasmas.} Draft in Progress, 2022.

% \item Ebrahimi F, Pharr M. {\sl A magnetic- and spatial-curvature driven instability in a differentially rotating plasma.} Draft Submitted to Astrophysical Journal, 2022.

% \item Pharr M, Ebrahimi F, Blackman E. {\sl Large Scale Magnetic Field Growth and Stability in Hall-MHD Simulations of Quasi-Keplerian Flows.}
% American Physical Society, Division of Plasma Physics 2021 Annual Meeting: Pittsburgh, PA. Section BO06.00003: Astrophysical Turbulence and Dynamos.

% \end{itemize}

Pharr M, Paz-Soldan C, Park J, Logan N, Gribov Y, Leuthold N. {\sl Error field source identification in early ITER plasmas.} American Physical Society, Division of Plasma Physics 2022 Annual Meeting: Spokane, Wa. Section PP11.00044: Poster Session VI: Diagnostics; Edge and Pedestal; Stability; Heating; Transport, Turbulence.

Pharr M, Ebrahimi F. {\sl Magneto-Curvature Instability in Hall-MHD plasmas.} Draft in Progress, 2022.

Ebrahimi, F., \& Pharr, M. (2022). {\sl A nonlocal magneto-curvature instability in a differentially rotating disk.} The Astrophysical Journal, 936(2), 145. \href{https://doi.org/10.3847/1538-4357/ac892d}{https://doi.org/10.3847/1538-4357/ac892d} 

Pharr M, Ebrahimi F, Blackman E. {\sl Large Scale Magnetic Field Growth and Stability in Hall-MHD Simulations of Quasi-Keplerian Flows.} American Physical Society, Division of Plasma Physics 2021 Annual Meeting: Pittsburgh, PA. Section BO06.00003: Astrophysical Turbulence and Dynamos.


\section{RESEARCH}
    {\sl Graduate Research Assistant} \hfill May 2021 - Present \\
    Columbia University Plasma Physics Lab, 
    New York, NY\\ 
    Supervisor: Dr. Carlos Paz-Soldan
    \begin{itemize}  \itemsep -2pt %reduce space between items
    \item Model 3D fields in different coil lead configurations for DIII-D using GPEC.
    \item Determine 3D field correction strategies for early ITER scenarios.
    \item Study linearity assumptions for error field perturbations on coil tilt and shift scenarios in NSTX-U. 
    \end{itemize}

    {\sl U.S. Department of Energy Research Intern} \hfill Fall 2020, Summer 2021 \\
    Princeton Plasma Physics Lab, 
    Princeton, NJ\\ 
    Supervisor: Dr. Fatima Ebrahimi
    \begin{itemize}  \itemsep -2pt %reduce space between items
    \item Investigate the relationship between Hall and MHD dynamo terms in the induction equation and the growth of large scale magnetic fields due to magnetorotational instability (MRI) using NIMROD.
    \item Write data analysis scripts for the PPPL MRI Experiment and model a new ideal MHD instability in differentially rotating plasmas (see second publication).
    \end{itemize}

    {\sl Undergraduate Research in Molecular Biophysics} \hfill Summer 2020, Spring 2021 \\
    Rensselaer Polytechnic Institute, 
    Department of Mathematical Sciences, Troy, NY\\
    Supervisor: Dr. Peter Kramer
    \begin{itemize}  \itemsep -2pt %reduce space between items
    \item Explore the efficacy of models for interactions between microtubule filaments using Monte Carlo methods.
    \item Implement Gillespie algorithm simulations for various models in python. 
    \item Expand and connect the two theores in unexplored parameter space.
    \end{itemize}

    {\sl Project in Applied Mathematics and Economics} \hfill Fall 2019 - Spring 2020
    \begin{itemize}  \itemsep -2pt %reduce space between items
        \item Investigate the time evolution of Lorentz curves for Monte-Carlo models of economies.
        \item Computationally implement/explore various wealth/income distribution models.
    \end{itemize}

    {\sl Mathematical Competition in Modeling} \hfill Spring 2020 \\
    Awarded Honorable Mention for Research Paper
    \begin{itemize}  \itemsep -2pt %reduce space between items
        \item Synthesize a model for the changing of migration patterns of Mackerel and Herring due to climate change.
        \item Study how this will affect Scottish fishing markets for different climate change outcomes.
        \item Write a research paper detailing the results of the study.
    \end{itemize}

\section{PEDAGOGY}
    
    {\sl Graduate Teaching Assistant} \hfill August 2022 - Present \\
    Barnard College, Physics and Astronomy, New York, NY
    \begin{itemize}  \itemsep -2pt %reduce space between items
        \item Teach introductory mechanics lab, grade lab reports.
    \end{itemize}

    {\sl Graduate Teaching Assistant} \hfill Fall 2021 - May 2022 \\
    Columbia University, Applied Physics and Applied Mathematics, New York, NY
    \begin{itemize}  \itemsep -2pt %reduce space between items
        \item Hold office hours for upper undergraduate/introductory graduate courses on Complex Analysis and Linear Algebra.
        \item Grade Homework Assignments. 
    \end{itemize}    


    {\sl Undergraduate Facilitator} \hfill Fall 2019 - Summer 2020, Spring 2021 \\
    Rensselaer Polytechnic Institute, Physics Department, Troy, NY
    \begin{itemize}  \itemsep -2pt %reduce space between items
        \item Facilitate Honors Physics I/II Labs, Electromagnetic Theory, Introduction to Quantum Mechanics (third course in QM sequence).
        \item Hold weekly office hours.
        \item Attend some courses to help answer questions.
    \end{itemize}

    {\sl I-PERSIST Mentor} \hfill Fall 2019 \\
    Rensselaer Polytechnic Institute, Physics Department, Troy, NY
    \begin{itemize}  \itemsep -2pt %reduce space between items
        \item Run two small weekly group meetings to strengthen important problem solving skills in Physics I.
        \item Guide 20 new freshmen to success in their first semester.
    \end{itemize}

    {\sl ALAC Tutor} \hfill Fall 2019 - Spring 2020 \\
    Rensselaer Polytechnic Institute, Physics Department, Troy, NY
    \begin{itemize}  \itemsep -2pt %reduce space between items
        \item Tutor Physics I, II, Honors Physics I, II students in private and group settings for Rensselaer's Learning Assistance Center.
    \end{itemize}

    {\sl Private Tutor} \hfill Spring 2018 - Spring 2020 \\
    Self-employed.
    \begin{itemize}  \itemsep -2pt %reduce space between items
        \item Synthesize lessons, example problems, and meet with math, computer science, and physics students regularly to assist with coursework and test preparation.
    \end{itemize}
    
    
\section{HONORS AND AWARDS} {\sl Rensselaer Polytechnic Institute}

                Max Hirsch Prize in Mathematics
                \begin{itemize}
                    \item[] {\sl This prize is awarded to a Senior in the Department of Mathematical Sciences who has demonstrated outstanding ability in his or her academic work and also gives promise of outstanding success in a career in mathematical sciences.}
                \end{itemize}
                \vspace{-0.4cm}
                J. Lawrence and Gertrude Katz Award in Physics
                \begin{itemize} 
                    \item[] {\sl This award is presented to the student selected as the outstanding graduating senior receiving a Bachelor of Science in Physics.}
                \end{itemize}
                \vspace{-0.4cm}
                $\Sigma \Pi \Sigma$, Physics Honor Society \\
                Rensselaer Archimedian Society (4.0 award) \\
                Honorable Mention for Research Paper,       Mathematical Competition in Modeling \\
                Rensselaer Leadership Award 
 
\section{SKILLS}
Python/Conda, Java, NumPy, SciPy, MatPlotLib, Linux Terminal, Matlab, \LaTeX\\
Society of Physics Students Build Team – Demos and Outreach \\
Use of vacuum technology and other plasma physics related lab equipment\\
Use of high voltage lab equipment\\
Conversational in French\\
Excellent English Reading/Writing Skills \\

 
\section{HOBBIES}
Hiking; Bouldering; Media Studies; Green, Queer, and Student Activism
 
\end{resume} 
\end{document}









