% LaTeX file for resume 
% This file uses the resume document class (res.cls)

\documentclass{rpires} 
%\usepackage{helvetica} % uses helvetica postscript font (download helvetica.sty)
%\usepackage{newcent}   % uses new century schoolbook postscript font 
\newsectionwidth{0pt}  % So the text is not indented under section headings
\usepackage{fancyhdr}  % use this package to get a 2 line header
\usepackage{hyperref}
\usepackage{xcolor}
\renewcommand{\headrulewidth}{0pt} % suppress line drawn by default by fancyhdr
\newcommand{\wideline}{\moveleft\hoffset\vbox{\hrule width\resumewidth height 1pt}\smallskip}
\setlength{\headsep}{12pt}  % space between header and text
\setlength{\headheight}{12pt} % allow room for 2-line header
% \setlength{\voffset}{1cm} % start text higher on the page
\setlength{\textheight}{9.5in} % increase text height to fit more on a page

\pagestyle{fancy}     % set pagestyle for document
\rhead{ {\it M. Pharr \it p. \thepage} } % put text in header (right side)
\cfoot{}                                     % the foot is empty
\topmargin=-0.5in % start text higher on the page

\begin{document}
\thispagestyle{empty} % this page has no header  
\name{\LARGE MATTHEW C. PHARR\\[12pt]}% the \\[12pt] adds a blank line after name

\begin{resume}
 
\wideline
\centerline{matthew.pharr@columbia.edu; (410) 375-9882}
\centerline{600 W 125th St, \#8D, New York, NY 10027, USA}
\vspace{-0.2cm}
\section{EDUCATION}
\vspace{0.3cm}\wideline
\vspace{-0.5cm}

\begin{center}
  \begin{tabular}{l l}
  {\textsl{Columbia University}}, New York, NY & \textsl{Rensselaer Polytechnic Institute}, Troy, NY\\
  Ph.D. Plasma Physics, Expected 2026  & B.S. Physics \&  Mathematics, 2021 \\
  M.Phil. Plasma Physics, 2025 & \sl Summa Cum Laude.\\
  M.S. Applied Physics, 2023 & {} 
  \end{tabular}
\end{center}

\section{PUBLICATIONS}
\vspace{0.3cm}\wideline
% \vspace{-0.5cm}

\begin{itemize}  \itemsep -2pt %reduce space between items
    \item {{\bf M. Pharr}, \textit{Quantifying and Mitigating Risk of Error Field Penetration in Non-Axisymmetric Tokamak Plasmas}, Ph.D. Dissertation, Columbia University (2026).}
    \item {{\bf M. Pharr}, N. C. Logan, J. K. Park, and C. Paz-Soldan, \textit{Metrics for quantifying resonant drive in ideal and extended MHD}, \href{}{Planned for submission 2025.}}
    \item {{\bf M. Pharr}, N. C. Logan, J. K. Park, and C. Paz-Soldan, \textit{Quantifying resonant drive in kinetically relaxed tokamak perturbed equilibria}, \href{}{Planned for submission 2025.}}
    \item {{\bf M. Pharr}, N. C. Logan, C. Paz-Soldan, J. K. Park, and C. Hansen, \textit{Error field predictability and consequences for ITER}, \href{https://dx.doi.org/10.1088/1741-4326/ad7ed6}{Nucl. Fusion 64, 126025 (2024)}.}
    \item {F. Ebrahimi and {\bf M. Pharr}, \textit{A Nonlocal Magneto-curvature Instability in a Differentially Rotating Disk}, \href{https://dx.doi.org/10.3847/1538-4357/ac892d}{ApJ 936, 145 (2022)}.}\vspace{0.5cm}
    \item {N. Logan et al., \textit{SPARC Tokamak Error Field Expectations and Physics-Based Correction Coil Design}, \href{}{Planned for submission to Nuclear Fusion, 2025.}}
    \item {X. Bai et al., \textit{Time variation of error field correction in ITER}, submitted to Plasma Physics and Controlled Fusion (2025).}
    \item {The MANTA Collaboration et al., \textit{MANTA: a negative-triangularity NASEM-compliant fusion pilot plant}, \href{https://iopscience.iop.org/article/10.1088/1361-6587/ad6708}{Plasma Phys. Control. Fusion 66, 105006 (2024)}.}
    \item {C. J. Hansen, I. G. Stewart, D. Burgess, M. Pharr, S. Guizzo, F. Logak, A. O. Nelson, and C. Paz-Soldan, \textit{TokaMaker: An open-source time-dependent Grad-Shafranov tool for the design and modeling of axisymmetric fusion devices}, \href{https://www.sciencedirect.com/science/article/pii/S0010465524000341}{Computer Physics Communications 298, 109111 (2024)}.}
    \item {C. Paz-Soldan et al., \textit{Simultaneous access to high normalized density, current, pressure, and confinement in strongly-shaped diverted negative triangularity plasmas}, \href{https://new.iopscience.iop.org/article/10.1088/1741-4326/ad69a4/meta}{Nucl. Fusion 64, 094002 (2024)}.}
    \item {C. T. Holcomb et al., \textit{DIII-D research to provide solutions for ITER and fusion energy}, \href{https://dx.doi.org/10.1088/1741-4326/ad2fe9}{Nucl. Fusion 64, 112003 (2024)}.}
    \item {S. Guizzo et al, \textit{Electromagnetic System Conceptual Design for a Negative Triangularity Tokamak}, \href{ https://doi.org/10.48550/arXiv.2501.14682}{ArXiv, submitted to Fusion Engineering and Design (2024)}.}
\end{itemize}

\section{PRESENTATIONS \& CONFERENCE PROCEEDINGS}
\vspace{0.3cm}\wideline
\begin{itemize}  \itemsep -2pt %reduce space between items
  \vspace{0.5cm}
    \item M. Pharr et al. \textit{Quantifying the Resonant Drive for Magnetic Islands in Perturbed Ideal, Resistive, and Kinetic MHD Equilibria.} American Physical Society, Division of Plasma Physics 2025 Annual Meeting: Long Beach, CA. Section GO05.00007: MFE: MHD.
    \item M. Pharr et al. \textit{Quantifying the Resonant Drive for Magnetic Islands in Perturbed Ideal, Resistive, and Kinetic MHD Equilibria.} 29th US-Japan Joint Workshop on MHD Stability Control: Princeton, NJ, 2025.
    \item M. Pharr et al. \textit{Expected Error Fields in ITER: A full-device source model and strategies for stable operation.} ITER International Summer School 2025: Aix-en-Provence, France.
    \item M. Pharr et al. \textit{Expected Error Fields in ITER: A full-device source model and strategies for stable operation.} American Physical Society, Division of Plasma Physics 2024 Annual Meeting: Atlanta, GA. Section PP11.00050: Research in Support of ITER.
    \item M. Pharr et al. \textit{Error Field Predictability and Consequences for ITER.} American Physical Society, Division of Plasma Physics 2023 Annual Meeting: Denver, Co. Section PP11.00050: Poster Session VI: MHD and Stability.
    \item M. Pharr, F. Ebrahimi, \textit{A nonlocal Curvature-Driven Flow-Shear Instability in Low-Field Plasmas} Sherwood Fusion Theory 2023 Conference: Knoxville, TN.
    \item M. Pharr et al. \textit{Error field source identification in early ITER plasmas.} American Physical Society, Division of Plasma Physics 2022 Annual Meeting: Spokane, Wa. Section PP11.00044: Poster Session VI: Diagnostics; Edge and Pedestal; Stability; Heating; Transport, Turbulence.
    \item M. Pharr, F. Ebrahimi, E. Blackman, \textit{Large Scale Magnetic Field Growth and Stability in Hall-MHD Simulations of Quasi-Keplerian Flows.} American Physical Society, Division of Plasma Physics 2021 Annual Meeting: Pittsburgh, PA. Section UO06.00014: Astrophysical Turbulence and Dynamos.
\end{itemize}

\section{HONORS \& AWARDS} 

\vspace{0.3cm}\wideline

\textit{ORFEAS Fusion Design Contest Award}; Columbia University \hfill 2022
\begin{itemize}
    \item[] {For contributions to the Columbia/MIT team's project on negative triangularity reactor pilot plant design scoping. Awarded \$22,000 as a group.}
\end{itemize}
\vspace{-0.4cm}

\textit{Max Hirsch Prize in Mathematics}; Rensselaer Polytechnic Institute \hfill 2021
\begin{itemize}
    \item[] {This prize is awarded to a Senior in the Department of Mathematical Sciences who has demonstrated outstanding ability in his or her academic work and also gives promise of outstanding success in a career in mathematical sciences.}
\end{itemize}
\vspace{-0.4cm}

\textit{J. Lawrence and Gertrude Katz Award in Physics}; Rensselaer Polytechnic Institute \hfill 2021
\begin{itemize} 
    \item[] {This award is presented to the student selected as the outstanding graduating senior receiving a Bachelor of Science in Physics.}
\end{itemize}
\vspace{-0.4cm}

$\Sigma \Pi \Sigma$, Undergraduate Physics Honor Society \hfill 2021 \\
Rensselaer Archimedian Society (4.0/4.0 GPA award)\hfill 2019 \\
Honorable Mention for Research Paper,       Mathematical Competition in Modeling \hfill 2020 \\
Rensselaer Leadership Award \hfill 2018


\section{RESEARCH \& PROJECTS}
\vspace{0.3cm}\wideline
\textit{Graduate Research Assistant, Columbia University} \hfill August 2021 - Present \\
Department of Applied Physics and Applied Mathematics, New York, NY \\
Advisor: Dr. Carlos Paz-Soldan, Dr. Nikolas Logan. 
\begin{itemize}
  \item {Research in theoretical and computational plasma physics, focusing 3D perturbations of tokamak equilibria, error field effects and its induced risk, and MHD stability.}
  \item {Created error field source model for ITER, and used it to predict the risk of error field penetration in ITER's baseline scenario. Used this model to propose a novel method for ensuring extrapolability of error field measurements from low to high beta.}
  \item {Contributions to the MANTA project, a joint Columbia/MIT negative triangularity pilot plant design study. Worked on integrated modeling of plasma equilibria, transport, and 0D systems modeling.}
  \item {Development and application of computational tools, including: \href{https://github.com/PrincetonUniversity/GPEC}{GPEC}, a fast linear perturbed equilibrium suite, including sub-modules DCON and its asymptotic matching code, RDCON/RMATCH, as well as an in-progress \href{https://github.com/OpenFUSIONToolkit/JPEC}{Julia-based modernization}; \href{https://github.com/OpenFUSIONToolkit/OpenFUSIONToolkit}{Tokamaker}, a flexible, open-source Grad-Shafranov solver for Tokamak and Dipole equilibria; \href{https://github.com/hansec/OpenPOPCON}{OpenPOPCON}, an open-source 0D tokamak systems code that has been used for scoping the MANTA and CENTUAR tokamak designs, as well as in consultation with fusion startups.}
  \item {Work for the SPARC MHD team, including maintaining and developing its total error field source model, giving feedback on iterative coil design, and predicting time-dependent risk of error field penetration. Developed a module for the \href{https://arxiv.org/html/2509.10244v1}{POPSIM code}, a plasma control system simulator, to include effects of time-dependent error fields and their driven tearing modes and penetration risks.}
  \item {Attended MIT CPS-FR Computational Summer School, 2023.}
\end{itemize}

\textit{SULI Intern, Princeton Plasma Physics Laboratory} \hfill Fall 2020 - Summer 2021 \\
Theory Department, Princeton, NJ \\
Supervisor: Dr. Fatima Ebrahimi.
\begin{itemize}
  \item {Studied large-scale magnetic field growth and stability in Hall-MHD simulations of quasi-Keplerian plasma flows, with applications to astrophysical accretion disks.}
  \item {Used the extended MHD code NIMROD to perform linear and nonlinear simulations of magnetized plasmas in cylindrical geometry.}
  \item {Discovered a new family of more global solutions to the flow-driven Magneto-Rotational Instability (MRI) through a higherarchy of models, from a new fast linear eigenvalue solver to full 3D nonlinear simulations on large-scale computing clusters.}
\end{itemize}

\textit{Undergraduate Research Assistant, Rensselaer Polytechnic Institute} \hfill Summer 2020 - Spring 2021 \\
Department of Mathematical Sciences, Troy, NY \\
Supervisors: Dr. Peter Kramer, Dr. Scott Forth.
\begin{itemize}
  \item {Studied the application of stochastic methods to the modeling of complex biological systems, including meiotic spindle assembly.}
  \item {Used python to implement a computational statistical model of microtubule dynamics, and performed data analysis and visualization.}
\end{itemize}

\section{PEDAGOGY}

\vspace{0.3cm}\wideline
    
    \textit{Graduate Teaching Assistant, Intro Physics Lab Sequence} \hfill Fall 2022 - Present \\
    Barnard College of Columbia University, Physics and Astronomy, New York, NY
    % \begin{itemize}  \itemsep -2pt %reduce space between items
    %     \item Teach introductory mechanics lab, grade lab reports.
    % \end{itemize}
    \vspace{-0.1cm}
    
    \textit{Senior Graduate Mentor, Fusion Reactor Design} \hfill Fall 2024 \\
    Columbia University, Applied Physics and Applied Mathematics, New York, NY
    \vspace{-0.1cm}

    \textit{Graduate Teaching Assistant, Complex Analysis/Linear Algebra} \hfill Fall 2021 - Spring 2022 \\
    Columbia University, Applied Physics and Applied Mathematics, New York, NY
    % \begin{itemize}  \itemsep -2pt %reduce space between items
    %     \item Hold office hours for upper undergraduate/introductory graduate courses on Complex Analysis and Linear Algebra.
    %     \item Grade Homework Assignments. 
    % \end{itemize}    
    \vspace{-0.1cm}

    \textit{Undergraduate Facilitator} \hfill Fall 2019 - Summer 2020, Spring 2021 \\
    Rensselaer Polytechnic Institute, Physics Department, Troy, NY
    % \begin{itemize}  \itemsep -2pt %reduce space between items
    %     \item Facilitate Honors Physics I/II Lab, Electromagnetic Theory, Intro to Quantum Mech.
    %     % \item Hold weekly office hours.
    %     % \item Attend some courses to help answer questions.
    % \end{itemize}
    \vspace{-0.1cm}

    \textit{I-PERSIST Mentor} \hfill Fall 2019 \\
    Rensselaer Polytechnic Institute, Physics Department, Troy, NY
    % \begin{itemize}  \itemsep -2pt %reduce space between items
    %     \item Run two small weekly group meetings to strengthen important problem solving skills in Physics I.
    %     \item Guide 20 new freshmen to success in their first semester.
    % \end{itemize}
    \vspace{-0.1cm}

    \textit{ALAC Introductory Physics Tutor} \hfill Fall 2019 - Spring 2020 \\
    Rensselaer Polytechnic Institute, Physics Department, Troy, NY
    % \begin{itemize}  \itemsep -2pt %reduce space between items
    %     \item Tutor Physics I, II, Honors Physics I, II students in private and group settings for Rensselaer's Learning Assistance Center.
    % \end{itemize}
    \vspace{-0.1cm}

    \textit{Private Tutor, Math/Physics} \hfill Spring 2018 - Spring 2020 \\
    Self-employed. Took 2-4 hours per week of university-level tutoring.
    % \begin{itemize}  \itemsep -2pt %reduce space between items
    %     \item Synthesize lessons, example problems, and meet with math, computer science, and physics students regularly to assist with coursework and test preparation.
    % \end{itemize}

\section{SKILLS}
\vspace{0.3cm}\wideline
Python, Julia, Fortran, Matlab, Linux, git, \LaTeX\\
OpenMP, MPI parallel computing\\
Application of Neural Networks and Machine Learning to Physics Problems\\
High Performance Computing \\
IT and Computer Networking \\
Use of vacuum technology and other plasma physics related lab equipment\\
Use of high voltage lab equipment\\
Native English, B2 French, Basic Italian, Mandarin\\
 
\end{resume} 

\end{document}













