% Using the RPI resume template found at
% https://www.rpi.edu/dept/arc/training/latex/resumes/

\documentclass[margin]{rpires} 
\usepackage{amsfonts}
\setlength{\textheight}{10in} 
\begin{document}  
% Center the name over the entire width of resume:
\moveleft.5\hoffset\centerline{\large\bf Matthew C. Pharr}
% Draw a horizontal line the whole width of resume:
 \moveleft\hoffset\vbox{\hrule width\resumewidth height 1pt}\smallskip
% address begins here
% Again, the address lines must be centered over entire width of resume:
 \moveleft.5\hoffset\centerline{pharrm@rpi.edu}
 \moveleft.5\hoffset\centerline{(410) 375-9882}
                        
\begin{resume}                        
 
% \section{OBJECTIVE}       Acceptance to PhD Programs in Plasma Physics and Applied Mathematics.
 
\section{EDUCATION}      

                Columbia University, New York, NY\\
                Ph.D. Applied Mathematics, Expected 2025 \\
                M.S. Applied Mathematics, Expected 2022 \\


                Rensselaer Polytechnic Institute, Troy, NY \hfill {\bf G.P.A. 4.0/4.0} \\
                B.S. Physics \&  Mathematics, 2021 \\

% \section{PUBLICATIONS} %Anticipation is real :) 

\section{RESEARCH}      
    {\sl SULI Undergraduate Research Intern} \hfill Fall 2020, Summer 2021 \\
    Princeton Plasma Physics Lab, 
    Princeton, NJ 
    \begin{itemize}  \itemsep -2pt %reduce space between items
    \item Full time undergraduate intern at PPPL in plasma and fusion sciences.
    \item   Investigate the relationship between Hall and MHD dynamo terms in the induction equation and the growth of large scale magnetic fields due to magnetorotational instability using computational tools such as NIMROD, under the supervision of Dr. Fatima Ebrahimi. 
    \item Summer 2021: work with the magnetorotational instability experiment group to do modeling for the experiment at PPPL.\\
    \end{itemize}

    {\sl Undergraduate Research in Molecular Biophysics} \hfill Summer 2020, Spring 2021 \\
    Rensselaer Polytechnic Institute, 
    Department of Mathematical Sciences, Troy, NY 
    \begin{itemize}  \itemsep -2pt %reduce space between items
    \item Explore the efficacy of two theoretical mathematical models for the interactions between microtubule filaments through connecting
    Kinesin-5 molecules using various Monte Carlo methods with Professor Peter Kramer.
    \item Implement Gillespie algorithm simulations for various models in python employing use of NumPy and other libraries. 
    \item   Expand and connect the two theores in unexplored parameter space.
    \end{itemize}

    {\sl Personal Project in Applied Mathematics and Economics} \hfill Fall 2019 - Spring 2020
    \begin{itemize}  \itemsep -2pt %reduce space between items
        \item Investigate the development of the Lorentz curve in a free market system for various Monte-Carlo models of an economy.
        \item Computationally implement and investigate various wealth and income distribution models.
    \end{itemize}

    {\sl Mathematical Competition in Modeling} \hfill Spring 2020 \\
    Awarded Honorable Mention for Research Paper
    \begin{itemize}  \itemsep -2pt %reduce space between items
        \item Formulate a model for the changing of migration patterns of Northern Atlantic Mackerel and Herring due to climate change.
        \item Study how this will affect the Scottish fishing market for different climate change outcomes.
        \item Produce recommendations for businesses to minimize negative impact in the format of a mathematical research paper.
    \end{itemize}
\newpage
\section{EXPERIENCE}
    {\sl Undergraduate Facilitator} \hfill Fall 2019 - Summer 2020, Spring 2021 \\
    Rensselaer Polytechnic Institute, Physics Department, Troy, NY
    \begin{itemize}  \itemsep -2pt %reduce space between items
        \item Facilitate Honors Physics I/II Labs, Electromagnetic Theory, Introduction to Quantum Mechanics (third course in QM sequence)
        \item Hold weekly office hours
        \item Attend some courses to help answer questions
    \end{itemize}

    {\sl I-PERSIST Mentor} \hfill Fall 2019 \\
    Rensselaer Polytechnic Institute, Physics Department, Troy, NY
    \begin{itemize}  \itemsep -2pt %reduce space between items
        \item Run two small weekly group meetings to strengthen important problem solving skills in Physics I 
        \item Guide 20 new freshmen to success in their first semester.
    \end{itemize}

    {\sl ALAC Tutor} \hfill Fall 2019 - Spring 2020 \\
    Rensselaer Polytechnic Institute, Physics Department, Troy, NY
    \begin{itemize}  \itemsep -2pt %reduce space between items
        \item Tutor Physics I,II Honors I, and Honors II students in private and group settings for Rensselaer's Learning Assistance Center
    \end{itemize}

    {\sl Private Tutor} \hfill Spring 2018 - Spring 2020 \\
    Self-employed.
    \begin{itemize}  \itemsep -2pt %reduce space between items
        \item Synthesize lessons, example problems, and meet with math, computer science, and physics students regularly to assist with coursework and test preparation.
    \end{itemize}
    
    
\section{HONORS} Rensselaer Archimedian Society (4.0 award) \\
                 Honorable Mention for Research Paper, Mathematical Competition in Modeling \\
                 Rensselaer Leadership Award 
 
\section{SKILLS}
Python/Conda, Java, NumPy, SciPy, Matplotlib, Linux Terminal, Matlab\\
Society of Physics Students Build Team – Demos and Outreach \\
Competent in handling lab equipment using basic shop machinery\\
RPI Putnam Team - Competition in mathematical problem solving\\
Conversational in French\\
Excellent English Reading/Writing Skills \\

 
\section{HOBBIES}         Taekwondo, Media Studies, Green and LGBTQ+ Activism
 
\end{resume} 
\end{document}









